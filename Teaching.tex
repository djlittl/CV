\documentclass[11pt]{article}
%\usepackage{courier}
\usepackage{times}
\newlength{\textwide} \setlength{\textwide}{6.5in}
\usepackage[total={\textwide,10.0in},top=1.0in,left=1.0in,headheight=0.0in,headsep=0.6in,footskip=0.0in]{geometry}
\usepackage{fancyhdr}
\setlength{\parindent}{0em}

\begin{document}

\thispagestyle{fancy}
\lhead[]{David Littlewood}
\rhead[]{January 2008}

% capstone projects?

% Be sure to study the institution's idea of what the learning experience should be like.  Big or small classes, etc.
% What is the student identity and background of this institution?  How does that effect how I will teach?  Teaching style should be dependent on audience, goals dependent on students (school, year).
% Schools that prepare for industry:  emphasize engineering as applied science, problem solving, use of proper tools.
% Schools that prepare for research:  mathematical theory more important?

% Checklist:
%
% Specific examples of how my ideas are played out in the mechanical engineering classroom
% Tone of humility and excitement
% Statement should state why, how, and what I teach (what I get out of teaching, goals, methods)
% My objectives and methods for reaching those objectives
% Balance of ideaological and practical

% Misc
%
% Lists seem to be a good idea

\begin{center}
{\Large \textbf{Teaching Philosophy and Interests}} \\
\end{center}
\vspace{.1in}

My teaching philosophy incorporates two core beliefs:  1)  effective communication between instructor and student is vital, and 2) learning in engineering is an iterative process of theoretical study and application.  It is through these principles that I reach my teaching goal---to provide an environment that allows students to take initiative in their education, and in which every student, regardless of background, comes away with an understanding of core mechanical engineering concepts.

\vspace{0.1in}

I am interested in teaching a broad range of engineering material and have particular interest in topics related to solid mechanics and computational methods.  My teaching is influenced by my research work in applied mechanics, plasticity, optimization, computer aided design, and the finite element method.  I have served as the instructor for Design and Analysis of Structures at Syracuse Univeristy and for Engineering Dynamics at Rensselaer Polytechnic Institute with very positive student feedback in both cases.

% incorporate

%The expectations that I place on students are dependent on their status as underclassmen, upperclassmen, or graduate students, and on their destination in industry or in a research setting.  My teaching philosophy is based on an examination of my own learning style and those of others, both from the viewpoint of an instructor and of a student.  These principles dictate the tools I use in teaching.

% book study
% theory conceptual instruction contemplation

\paragraph{Communication.}  As an instructor, I am responsible for the clear and deliberate communication of course material.  This begins with the identification of course goals, taking into account student status, background, and destination.  Next is perhaps the most important stage, the thoughtful preparation of course materials.  My experience has shown that careful and thorough preparation is the most important effort I can make toward a successful class experience.  This involves the distillation of material from a broad set of sources into a well organized lecture emphasizing key concepts.  Well organized must not be mistaken for rigid, however.  The ability to recognize the underlying nature of students' questions, and to respond accordingly, is vital to effective teaching.  An example from my own teaching experience is the presentation of tensor invariants.  Students who struggled with this concept frequently asked questions regarding mathematical details, but it was clear to me that they were failing to grasp the key underlying concept.  These students benefited when invariants were presented in terms of the basic physical principle that the underlying states of stress and strain in a material cannot change as a result of rotating the coordinate system.

%\begin{itemize}
%\item Define needs of students - Identify the need of the student - is this a core area for them that they must understand in detail, or do then mainly just need broad concepts?
%\item Instructor is responsible for identifying key concepts and presenting them clearly and deliberately (particularly to students who aren't that interested).
%\item A key to good communication is careful organization of ideas and intense lecture preparation.
%\item Students are responsible for communicating needs.
%\item Instructor is responsible for listening to and responding to needs.
%\item Example of relating math to common-sense ideas: stress/strain tensor invariants, math can be daunting but the idea that frame of reference should not change the underlying stress/strain state is basic.
%\item Understanding students questions and identifying the concept that they're missing.  Do not continue to talk over their heads.  Are they asking a detail question, or are they clearly missing a main concept?
%\item Communication can be helped with modern teaching tools used in moderation, online notes, etc.  (concrete example, Felippa notes and mathematica)
%\end{itemize}

\paragraph{Learning as an iterative process.}  My lecture and exercise preparation is driven by the idea that learning in engineering is an iterative process of theoretical study and hands-on application.  My own understanding of engineering material stems from the cyclic process of expanding on theory in the literature and applying it to problems in computational mechanics.  Modern finite element software provides an example of how this process can be applied to teaching; theoretical concepts from the classroom come to life when students apply them to solve problems with these software tools.  I intend to bridge my teaching and research when appropriate in accordance with this approach.  Seeing state-of-the-art engineering tools being applied to solve real-world problems can foster interest in a subject.  In academia, as in industry, students should ultimately be evaluated based on their creative ability to apply concepts for the solution of engineering problems.  An iterative approach to teaching cultivates these talents and develops the fundamental skills required for competent engineering.

%\begin{itemize}
%\item My own experience in computational solid mechanics - read, code, read, code.
%\item I learn the most when I'm interested, and as an engineer, the application of concepts to solve problems is interesting.
%\item Foster interest in subject by demonstrating how it is applied to solve problems and how it ties in with other material.
%\item Example:  demonstrate how FEM software is an implementation of core ideas.
%\item Hope to relate to research by developing hands-on exercises derived from research.
%\item Engineering students should be ultimately evaluated based on their ability to apply concepts to arrive at problem solutions - this fits in with iterative model.
%\end{itemize}

%\paragraph{The rewards of teaching.}  Teaching offers me the opportunity for fulfilling interaction with students, and compels me to think critically about subject matter in a way that I have not experienced elsewhere.  I place great value on my interactions with others, both students and faculty.  The opportunity to guide students through a pivotal stage in their lives is a primary reason why I have chosen an academic career.  Students have given me a high overall rating and have listed my ability to present material in an accessible manner as one of my key strengths.  I look forward to applying myself to course and curriculum development with fellow faculty in mechanical engineering.

% Willing and able to teach almost any undergraduate course
% Graduate courses in FEM, solids, computational methods.

\paragraph{Specific teaching interests.}  I would be glad to teach a broad range of topics in mechanical engineering, with a general focus in solid mechanics and computational methods.  Examples of specific undergraduate material include statics, dynamics, strength of materials, engineering design, and computational methods.  I am also interested in introducing new students to engineering through classes such as Introduction to Mechanical Engineering and Introduction to Engineering Computing.  My interests in teaching at the graduate level are tied to my research work in applied mechanics.  Specific graduate-level teaching interests include solid mechanics, optimization, finite element analysis, and parallel-computing methods.

% Courses:
%
% Applied mechanics
% Design
% Laboratory
% continuum mechanics
% mechanics of materials
% strength of materials
% mechanics of solids
% parallel computing
% computational mechanics
%
% GEEN 1300 Introduction to Engineering Computing	
% MCEN 1000 Introduction to Mechanical Engineering
% GEEN 1400 First-year Engineering Projects
% MCEN 2023 Statics and Structures
% MCEN 2024 Materials Science
% MCEN 2063 Mechanics of Solids
% MCEN 3025 Component Design
% MCEN 3030 Computational Methods
%
% ENGR-1100 Intro. to Engineering Analysis
% ENGR-1300 Engineering Processes
% ENGR-1600 Materials Science for Engineers
% ENGR-2530 Strength of Materials
% ENGR-2050 Intro. to Engineering Design
% ENGR-2090 Engineering Dynamics
% CSCI-1190 Beginning C Programming for Engineers
% MANE-4670 Mechanical Behavior of Materials I 
% MANE-4280 Design Optimization

\end{document}
