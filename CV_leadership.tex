
\vspace{\sectionskip}
\noindent
{\large \textbf{PROJECT LEADERSHIP EXPERIENCE}}
\vspace{\sectionskip}

\begin{minipage}{\minipagewidth}
\textbf{Project Manager}, ASC ATDM (2019 - Present) \\
%
Programmatic lead for the ASC ATDM Algorithms and Applications effort, including the EMPIRE Verification and Validation, EMPIRE-Fluid, SPARTA DSMC, and Technology Demonstrator projects.  Responsible for coordination of planning, execution, and reporting activities among multiple Principal Investigators and stakeholders.
\end{minipage}\vspace{\parskip}

\begin{minipage}{\minipagewidth}
\textbf{Sandia Principal Investigator}, DOE SBIR (2019 - Present) \\
%
Sandia PI for the DOE SBIR project ``A Multi-Physics Analysis Capability for Engine Materials,'' in partnership with Sundergo Labs, Inc.  Contributed to proposal writing and reporting, and served as the technical lead for the development of advanced peridynamic capabilities for modeling the effects of chemical diffusion on mechanical integrity.
\end{minipage}\vspace{\parskip}

\begin{minipage}{\minipagewidth}
\textbf{Sandia Principal Investigator} DOE STTR (2019 - Present) \\
%
Sandia PI for the DOE STTR project ``A Mesh Free Framework for Mechanical Simulations of Microstructure Data Files,'' in partnership with CFD Research Corporation.  Led the C++ implementation effort for a GPU-compatible peridynamics code utilizing the Kokkos library.  Partnered with the CFD PI for proposal writing, project execution, and reporting.
\end{minipage}\vspace{\parskip}

\begin{minipage}{\minipagewidth}
\textbf{Principal Investigator} ASC ATDM (2016 - Present) \\
%
Led the ``Multiscale Technology Demonstrator'' project to modernize Sandia engineering codes for next-generation computer architectures.  Coordinated efforts among Center 1400, 1500, and 8700 for the demonstration of emerging ATDM technologies in the NimbleSM code.  Responsible for project planning, execution, and reporting.  Contributed to the ASC ATDM FY17 L2 Milestone ``Asynchronous Many-Task Software Stack Demonstration.''
\end{minipage}\vspace{\parskip}

\begin{minipage}{\minipagewidth}
\textbf{Sandia Principal Investigator} DOE SBIR (2017 - 2018) \\
%
Sandia PI for the DOE SBIR project ``Hardening of the Peridigm Peridynamics Software for Industry Use,'' with Sundergo Labs, Inc.  Collaborated with the Sundergo PI for the prototype implementation of a peridynamics-FEM coupling strategy, including planning, proposal writing, project execution, and reporting.
\end{minipage}\vspace{\parskip}

\begin{minipage}{\minipagewidth}
\textbf{Principal Investigator} ASC PEM (2015 - Present) \\
%
Principal Investigator for the ASC PEM project ``Next-Generation Material Models.''  Responsible for planning, execution, and reporting on development of a LAME API redesign for Kokkos compatibility.  Successfully completed the ASC PEM 2017 L3 Milestone ``Adapting Material Models for Improved Performance on Next-Generation Hardware.''
\end{minipage}\vspace{\parskip}

\begin{minipage}{\minipagewidth}
\textbf{Principal Investigator} CIS LDRD (2013 - 2016) \\
%
Principal Investigator for the CIS LDRD project ``Strong Local-Nonlocal Coupling for Integrated Fracture Modeling.''  Developed novel techniques for coupling classical local models and nonlocal peridynamic models for solid mechanics.  Directed work of external collaborator Dr.~Pablo Seleson.  Responsible for proposal writing, project execution, and reporting.
\end{minipage}\vspace{\parskip}

\begin{minipage}{\minipagewidth}
\textbf{Task Lead} DoD DOE Joint Munitions Program (2013 - 2016) \\
%
Served as the Task Lead for Joint Munitions Program TCG-XI Task 3.2:  Perforation of Multiple Concrete Layers.  Provided leadership in the application of peridynamics to the simulation of material failure and fragmentation for problems of interest to the DOE and DoD.  Participated in planning, execution, and reporting to DoD customers.
\end{minipage}\vspace{\parskip}

