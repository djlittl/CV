\documentclass[11pt]{article}

% standard fonts:  helvet, avant, courier, chancery, times, palatino, bookman, newcent, charter
% command for setting default font family:  \renewcommand{\familydefault}{\sfdefault}
\usepackage{times}
\usepackage[square,numbers]{natbib}
\usepackage[bookmarks, bookmarksnumbered, colorlinks=true, 
            plainpages=false, citecolor=green, linkcolor=blue,
            urlcolor=blue, filecolor=blue]{hyperref}



\newlength{\textwide} \setlength{\textwide}{6.5in}
\usepackage[total={\textwide,10.0in},top=0.7in,bottom=01.3in,left=1.0in,headheight=0.0in,headsep=0.5in,footskip=0.3in]{geometry}
\usepackage{fancyhdr}

\pagestyle{myheadings}
\markboth{David Littlewood}{David Littlewood}

\newlength{\itemskip} \setlength{\itemskip}{0.0in}
\newlength{\sectionskip} \setlength{\sectionskip}{0.2in}
\newlength{\listskip} \setlength{\listskip}{0.05in}
\newlength{\minipagewidth} \setlength{\minipagewidth}{6.25in} % was 6.35in and 0.15in
\newlength{\bulletminipagewidth} \setlength{\bulletminipagewidth}{6.05in} % fudged
\newlength{\myparindent} \setlength{\myparindent}{0.25in}
\newlength{\myheadheight} \setlength{\myheadheight}{0.24in}
\newlength{\mytopmargin} \setlength{\mytopmargin}{-0.5in}
\setlength{\parindent}{\myparindent}
\setlength{\parskip}{.06in} % also need to set this inside \parbox

% bibliography options:
% create simple numbered list
\makeatletter
\renewcommand\@biblabel[1]{#1.}
\makeatother
% doi links
%\newcommand*{\doi}[1]{\href{http://dx.doi.org/#1}{doi: #1}}
\newcommand*{\doi}[1]{\href{http://dx.doi.org/#1}{doi}}
% bibunits for multiple bibliographies
\usepackage{bibunits}

% found this on the web to cause reverse numbering of bib items.  Incredible.
\usepackage{etaremune, etoolbox}
\makeatletter
\AtBeginDocument{%%% natbib redefines the environment there
\renewenvironment{thebibliography}[1]
 {\bibsection\parindent\z@\bibpreamble\bibfont
  \settowidth{\dimen0}{#1.}%
  \setlength{\dimen2}{\dimen0}%
  \addtolength{\dimen2}{\labelsep}
  \begin{etaremune}[labelwidth=\dimen0,leftmargin=\dimen2]
  \ifNAT@openbib
    \renewcommand\newblock{\par}%
  \else
    \renewcommand\newblock{\hskip.11em \@plus .33em \@minus .07em}%
  \fi
  \sloppy
  \clubpenalty4000
  \widowpenalty4000
  \sfcode`\.\@m
  \let\NAT@bibitem@first@sw\@firstoftwo
  \let\citeN\cite\let\shortcite\cite\let\citeasnoun\cite}
 {\bibitem@fin\bibpostamble
  \def\@noitemerr{\PackageWarning{natbib}{Empty `thebibliography' environment}}%
  \end{etaremune}
  \bibcleanup}
}%%% end of \AtBeginDocument

%%% patch \@lbibitem to use only \item (for etaremune)
\patchcmd{\@lbibitem}{\item[\hfil\NAT@anchor{#2}{\NAT@num}]}{\item}{}{}
\makeatother

\begin{document}

\thispagestyle{empty}

\begin{center}
{\LARGE \textbf{David Littlewood}} \\
\vspace{0.1in}
%
%Sandia National Laboratories \\
%Multiscale Science (Org.~1444), MS 1322 \\
%P.O. Box 5800, Albuquerque, NM 87185 \\
%djlittl@sandia.gov \hspace{.1in} (505) 284-0830 \\
%\href{http://www.sandia.gov/~djlittl/}{www.sandia.gov/$\sim$djlittl/} \\
9404 Admiral Nimitz NE \\
Albuquerque, NM 87111 \\
david.littlewood@gmail.com \\
505-340-6824 \\
\vspace{0.3in}
%
\end{center}

\noindent
{\large \textbf{EDUCATION}}
\vspace{\sectionskip}

\begin{minipage}{\minipagewidth}
\textbf{Ph.D., Mechanical Engineering}, University of Colorado at Boulder, December 2001 %({\small GPA 3.8/4.0})\\
%{\it Dissertation}: Printer Color Management Using Pareto-Optimization and Efficient Calibration \\
%Techniques \\

\vspace{\listskip}
{\it Advisor}: Professor Ganesh Subbarayan
\end{minipage}\vspace{\parskip}

\begin{minipage}{\minipagewidth}
\textbf{M.S., Mechanical Engineering}, University of Colorado at Boulder, May 1999 %({\small GPA 3.8/4.0})
\end{minipage}\vspace{\itemskip}

\begin{minipage}{\minipagewidth}
\textbf{B.S., Mechanical Engineering}, University of Colorado at Boulder, May 1995 %({\small GPA 3.2/4.0})
\end{minipage}

\vspace{\sectionskip}
\noindent
{\large \textbf{RESEARCH INTERESTS}}
\vspace{\sectionskip}

\begin{minipage}{\minipagewidth}
\setlength{\parindent}{\myparindent}
\noindent \textbf{Computational Solid Mechanics} \\
\indent Finite element methods \\
\indent Constitutive modeling \\
\indent Peridynamics
\end{minipage}\vspace{\parskip}

\begin{minipage}{\minipagewidth}
\setlength{\parindent}{\myparindent}
\noindent \textbf{Scientific Computing} \\
\indent Algorithm development for massively parallel systems \\
\indent Heterogeneous next-generation platforms \\
\indent Optimization
\end{minipage}

% \indent Imaging, computer graphics, and visualization \\

\vspace{\sectionskip}
\noindent
{\large \textbf{EMPLOYMENT HISTORY}}
\vspace{\sectionskip}

\begin{minipage}{\minipagewidth}
\textbf{Principal Member of the Technical Staff}, Sandia National Laboratories (Jan.~2014 - Present)
\vspace{0.06in}

I perform research in the formulation, implementation, and application of multiscale, multiphysics modeling techniques for computational solid mechanics.
\vspace{0.06in}

(1)~~Principal Investigator for the ASC-ATDM/ECP Multiscale Technology Demonstrator project.  The goal of this project is to drive development of emerging technologies for the Exascale Computing Project (ECP) and the Advanced Simulation and Computing (ASC) Advanced Technology Development and Mitigation (ATDM) program.  Key elements include the \emph{Kokkos} package for performance portability across disparate hardware architectures and the \emph{DARMA} asynchronous many-task scheduler.
\vspace{0.06in}

%(1)~~Principal Investigator of the Laboratory Directed Research and Development (LDRD) project ``Strong Local-Nonlocal Coupling for Integrated Fracture Modeling.''  The goal of this project is development of a mathematically consistent formulation of local-nonlocal coupling that allows for full integration of peridynamics with classical finite element analysis.  An open-source, collaborative software framework provides a proving ground for candidate approaches. Deployment to the laboratory will occur through the \emph{Sierra/SolidMechanics} production code.
%\vspace{0.06in}

(2)~~Serving as a lead developer for the open-source peridynamics code \emph{Peridigm}.  This work employs Sandia's \emph{Trilinos} software suite, including the \emph{Sacado} automatic-differentiation package for construction of the tangent matrix for implicit time integration.  Current applications include the modeling of ductile material failure and simulation of material fracture under blast loading conditions.
\vspace{0.06in}

(3)~~Developing a crystal plasticity constitutive model suitable for multiscale, multiphysics modeling within the \emph{Albany/LCM} simulation code.  Multiphysics modeling will capture the effect of hydrogen embrittlement.  A concurrent, Schwarz-based coupling scheme will allow the linking of an explicitly-modeled grain structure at the mesoscale with a continuum plasticity model at the component scale.
\vspace{0.06in}

\end{minipage}\vspace{\parskip}

%\vspace*{\fill}
\newpage
\pagestyle{fancy}
\lhead[]{David Littlewood}
\rhead[]{September 2017, p.~\thepage}
\cfoot{}
\setlength{\headheight}{\myheadheight}
\setlength{\topmargin}{\mytopmargin}


\begin{minipage}{\minipagewidth}
\textbf{Senior Member of the Technical Staff}, Sandia National Laboratories (Oct.~2008 - Jan.~2014) \\ 
%
Performed research and development in computational simulation and high-performance computing.  Focused on the development of theory, algorithms, and software for the application of peridynamics to solid mechanics problems involving material damage and failure.  Strengthened collaboration between Sandia's Computing Research and Engineering Science centers.  Served as technical lead for collaborative effort with Professor J.-S.~Chen for the implementation of RKPM in Sandia analysis codes.
\end{minipage}\vspace{\parskip}

\begin{minipage}{\minipagewidth}
\textbf{Research Associate}, Rensselaer Polytechnic Institute (Jun.~2006 - Oct.~2008) \\ 
%
Completed research in the area of computational mechanics in collaboration with Professor Antoinette Maniatty.  Implemented a multiscale framework for modeling the response of polycrystalline materials.  Developed models for capturing the incubation and nucleation stages of microstructurally small fatigue cracks in Al7075.  Contributed to the development of research proposals that were awarded \$550,000 in funding from DOD, NSF, and New York State.  Mentored students and worked closely with industrial and academic partners.
\end{minipage}\vspace{\parskip}

\begin{minipage}{\minipagewidth}
\textbf{Post-Doctoral Research Associate}, Rensselaer Polytechnic Institute (Feb.~2004 - Jun.~2006) \\ 
%
Developed a crystal plasticity constitutive model and associated finite element formulation for the DARPA Structural Integrity Prognosis System project under the mentorship of Professor Antoinette Maniatty.  Modeled Al7075 using three-dimensional, explicitly discretized grain structures.  Implemented finite element software in C++ for use on large-scale parallel computers (e.g., linux clusters and IBM BlueGene supercomputers).  Programming work was done in collaboration with the Cornell Fracture Group, led by Professor Anthony Ingraffea.
\end{minipage}\vspace{\parskip}

\begin{minipage}{\minipagewidth}
\textbf{Adjunct Faculty}, Rensselaer Polytechnic Institute (Jan.~2006 - May 2006) \\ 
%
Taught Engineering Dynamics as an adjunct professor in the Department of Mechanical, Aerospace, and Nuclear Engineering.  Engineering Dynamics is a mandatory subject in the core curriculum, requiring instructors to coordinate over multiple sections.  Course content included kinematics, kinetics, energy and momentum methods, and vibrations.  Received an overall instructor rating of 4.6/5.0. 
\end{minipage}\vspace{\parskip}

% Course content includes kinematics and kinetics of translational, rotational, and general plane motion, energy and momentum methods, and single-degree-of-freedom vibrations.

\begin{minipage}{\minipagewidth}
\textbf{Adjunct Assistant Professor}, Syracuse University (Jan.~2003 - May 2003) \\ 
%
Served as an adjunct professor in the Department of Mechanical, Aerospace, and Manufacturing Engineering.  Taught Design and Analysis of Structures, a second course in solid mechanics.  Course material included general principles of stress and strain, elasticity, and energy methods.  Received an overall instructor rating of 4.5/5.0.
\end{minipage}\vspace{\parskip}

% General methods were applied to solve problems of torsion, beam bending, and thick-wall cylinder deformation.

\begin{minipage}{\minipagewidth}
\textbf{Application Developer}, Glottal Enterprises, Syracuse, New York (Dec.~2002 - Jan.~2004) \\ 
%
Developed the SpeechTutor software package for the Glottal OroNasal
product line, a speech-training system for the hearing impaired.  The
SpeechTutor package processes speech input and provides graphical
feedback relating to the nasality and pitch aspects of speech.
Responsible for all aspects of software development, including project
specifications, program architecture, and programming.
\end{minipage}\vspace{\parskip}

\begin{minipage}{\minipagewidth}
\textbf{Color Scientist}, Quark, Inc., Denver, Colorado (May 2001 - Jul.~2002) \\ 
%
Solved color science problems for Quark software development while
evaluating new technologies as a member of the Quark Research
Lab. Prototyped advanced color functionality in QuarkXPress and the
Quark CMS XTension. New technologies, including image processing and
client/server applications, were implemented using C, C++, SVG, and
JavaScript.
\end{minipage}\vspace{\parskip}

%\begin{minipage}{\minipagewidth}
%\textbf{Research Assistant}, University of Colorado at Boulder (Dec.~1997 - Dec.~2001) \\
%%
%Completed work in Professor Ganesh Subbarayan's research group in the
%areas of color conversion and color printer characterization. This
%project was done in conjunction with IBM and involved the application
%of artificial neural networks, interpolation schemes, and optimization
%techniques to the problem of conversion between color spaces.
%% This project involved extensive C++ programming.
%\end{minipage}\vspace{\parskip}

%\noindent
%\textbf{Tutor}, University of Colorado at Boulder Athletic Department (Aug. 1997 - Dec. 1997) \\
%Tutored student athletes through the University of Colorado Athletic
%Department.  Covered a variety of topics including calculus, physics,
%and mechanics.  Mentored struggling students individually, and
%provided general assistance available to all student athletes.

\begin{minipage}{\minipagewidth}
\textbf{Project Engineer}, BNP Associates, Inc., Aurora, Colorado (Oct.~1995 - Jul.~1997) \\
%
Employed at an airline-industry consulting firm with both field and
office duties. Office work included technical writing and drawing
preparation for bid packages and specifications.  Fieldwork entailed
testing newly installed conveyor systems and computer
equipment. Frequently interacted with clients regarding both bid
packages and field installations.
\end{minipage}\vspace{\parskip}

%\vspace{\sectionskip}
%\noindent
%{\large \textbf{WORK IN PROGRESS}} 
%\vspace{\sectionskip}
%
%\begin{minipage}{\minipagewidth}
%\textbullet \hspace{0.1in}
%\begin{minipage}[t]{\bulletminipagewidth}
%Multi-Scale Modeling of Al7075 Using Crystal Plasticity and Periodic Grain Structures.
%\end{minipage}
%\end{minipage}\vspace{\parskip}
%
%\begin{minipage}{\minipagewidth}
%\textbullet \hspace{0.1in}
%\begin{minipage}[t]{\bulletminipagewidth}
%Identification of Critical Slip Planes with Crystal Plasticity and Visualization Techniques.
%\end{minipage}
%\end{minipage}

%\vspace{\sectionskip}

\begin{bibunit}[unsrtnat]
  \renewcommand{\refname}{{\large PEER-REVIEWED JOURNAL ARTICLES}}
  \nocite{*}
  \putbib[CV_journal_articles]
\end{bibunit}

\vspace{-0.1in}

\begin{bibunit}[unsrtnat]
  \renewcommand{\refname}{{\large BOOK CHAPTERS}}
  \nocite{*}
  \putbib[CV_book_chapters]
\end{bibunit}

\vspace{-0.1in}

\begin{bibunit}[unsrtnat]
  \renewcommand{\refname}{{\large CONFERENCE PROCEEDINGS}}
  \nocite{*}
  \putbib[CV_conference_proceedings]
\end{bibunit}

\vspace{-0.1in}

\begin{bibunit}[unsrtnat]
  \renewcommand{\refname}{{\large TECHNICAL REPORTS}}
  \nocite{*}
  \putbib[CV_tech_reports]
\end{bibunit}

%\input{CV_presentations.tex}

\vspace{-0.1in}

\begin{bibunit}[unsrtnat]
  \renewcommand{\refname}{{\large COURSES TAUGHT}}
  \nocite{*}
  \putbib[CV_courses_taught]
\end{bibunit}

\vspace{-0.1in}

\begin{bibunit}[unsrtnat]
  \renewcommand{\refname}{{\large MINISYMPOSIA ORGANIZED}}
  \nocite{*}
  \putbib[CV_symposia]
\end{bibunit}

\vspace{-0.1in}

\begin{bibunit}[unsrtnat]
  \renewcommand{\refname}{{\large STUDENTS MENTORED}}
  \nocite{*}
  \putbib[CV_students]
\end{bibunit}

\vspace{-0.1in}

\begin{bibunit}[unsrtnat]
  \renewcommand{\refname}{{\large JOURNAL REFEREE}}
  \nocite{*}
  \putbib[CV_referee]
\end{bibunit}

%\input{CV_references.tex}

%\vspace{\sectionskip}
%\noindent
%{\large \textbf{REFERENCES}}
%\vspace{\sectionskip}
%
%References available upon request.

\vspace*{\fill}

\end{document}
